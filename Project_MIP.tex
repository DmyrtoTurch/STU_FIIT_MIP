\documentclass[a4paper,12pt]{article}
\usepackage[utf8]{inputenc}
\usepackage[T1]{fontenc}
\usepackage[slovak]{babel}
\usepackage{setspace}
\usepackage{graphicx}
\usepackage{titlesec}
\setstretch{1.3}

\begin{document}

\begin{center}
    \textbf{Mykhailo Tereshchenko, Mykola Tyryk, Dmytro Turchanenko, Yekateryna Tiulkina}
\end{center}

\noindent\rule{\textwidth}{0.4pt}

\vspace{0.5em}

\begin{center}
    \LARGE \textbf{Rozšírená realita pre interaktívne vzdelávanie}
\end{center}

\begin{center}
    \large \textbf{EduAR}
\end{center}

\vspace{1em}

\section{Úvod do problematiky}

Súčasné vzdelávacie programy čelia viacerým problémom. Napríklad: angažovanosť študentov a preťaženie. V našej digitálnej ére metódy odovzdávania vedomostí zastarali, najmä v disciplínach STEM.

Problémom nie je nedostatok záujmu, ale fakt, že naša pamäť má obmedzené zdroje. Keď sa komplexné trojrozmerné objekty študujú pomocou dvojrozmerných schém a obrázkov, náš mozog stráca veľkú časť svojich zdrojov na vizualizáciu týchto objektov.

Zníženie motivácie teda nie je dôsledkom nudy, ale príznakmi kognitívnej únavy. A základ problému spočíva v zastaranosti prezentácie informácií, a nie v zložitosti materiálov.

Preto navrhujeme projekt EduAR, ktorý je zameraný na riešenie tohto problému, ponúkajúc technológiu, ktorá prezentuje informácie v prirodzenej trojrozmernej, interaktívnej forme, aby uvoľnila mentálne zdroje pre lepšie osvojenie si učiva.

\vspace{1em}
\section{Ciele projektu}

Hlavnými cieľmi projektu sú:

\begin{itemize}
    \item Vytvoriť vzdelávací systém s názvom EduAR, ktorý využíva technológiu rozšírenej reality na zlepšenie efektivity učenia.
    \item Znížiť pracovnú záťaž študentov a uvoľniť ich mentálne zdroje s cieľom zlepšiť a zjednodušiť pochopenie učiva.
    \item Zvýšiť motiváciu študentov tým, že sa proces učenia stane zaujímavejším a aktívnejším.
    \item Vyvinúť technológiu rozšírenej reality a implementovať ju vo vzdelávacích inštitúciách.
    \item Otestovať účinnosť rozšírenej reality prostredníctvom prieskumov a spätnej väzby od učiteľov a študentov.
\end{itemize}

\vspace{1em}
\section{Výhody}
\begin{itemize}
    \item Viac motivácie k učeniu vďaka využívaniu moderných technológií.
    \item Ľahšie porozumenie vďaka zobrazeniu zložitých systémov a objektov v 3D priestore.
    \item Študenti so zdravotným postihnutím sa môžu lepšie učiť s pomocou AR technológie.
\end{itemize}

\section{AR in study}

Rozšírená realita (Augmented Reality – AR) predstavuje technológiu, ktorá umožňuje spájať virtuálne prvky s reálnym prostredím. V oblasti vzdelávania otvára nové možnosti pre interaktívne a zážitkové učenie. Študenti môžu napríklad prostredníctvom smartfónov alebo špeciálnych okuliarov vidieť 3D modely buniek, molekúl, historických artefaktov či technických mechanizmov priamo v triede alebo doma.

Takýto prístup umožňuje hlbšie pochopenie učiva, keďže študenti nie sú len pasívnymi pozorovateľmi, ale aktívne skúmajú objekty z rôznych uhlov a v reálnom čase. AR tiež podporuje kolaboratívne učenie – študenti môžu spolupracovať pri riešení úloh v zdieľanom virtuálnom priestore.

Vo vedeckých disciplínach, ako je biológia, chémia, fyzika či technika, umožňuje AR bezpečné experimentovanie bez potreby drahých materiálov alebo laboratórií. Tým sa vzdelávací proces stáva prístupnejším, interaktívnejším a efektívnejším.

\section{TODO: REMOVE LATER}

\begin{figure}
    \centering
    \includegraphics[width=0.5\linewidth]{images/image.jpg}
    \label{fig:placeholder}
\end{figure}

\end{document}