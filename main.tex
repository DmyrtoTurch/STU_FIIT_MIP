% старт документу
\documentclass[a4paper,12pt]{article}
% тут у нас головні пакети

\usepackage[utf8]{inputenc}
\usepackage[T1]{fontenc}
\usepackage[slovak]{babel}
% це для форматування
\usepackage{setspace}
\usepackage{titlesec}
\setstretch{1.3}

\begin{document}

% тут імена
\begin{center}
    \textbf{Mykhailo Tereshchenko, Mykola Tyryk, Dmytro Turchanenko, Yekateryna Tiulkina}
\end{center}

\noindent\rule{\textwidth}{0.4pt} % це у нас чорна лінія

\vspace{0.5em}

% назва проетку
\begin{center}
    \LARGE \textbf{Rozšírená realita pre interaktívne vzdelávanie}
\end{center}

% це код для відступа
%\vspace{0.1em}

% акронім
\begin{center}
    \large \textbf{EduAR}
\end{center}

\vspace{1em}

% тут у нас має бути текст про проблеми які вирішує наш проєкт
\section{Úvod do problematiky}

% 2000 знаків (≈ 300–350 slov)
Tu napíš text o úvode, problematike a cieľoch projektu.  
Tento priestor je určený pre detailný opis, prečo je projekt potrebný,  
aký problém rieši a aké sú jeho hlavné ciele.

\vspace{1em}
% тут наші цілі проєкту мають бути
\section{Ciele projektu}

Tu opíš,

\vspace{1em}
% тут треба написати що нас проєкт принесе людям
\section{Výhody}

Tu opíš, aké prínosy projekt prinesie — napríklad zlepšenie efektivity,  
nové poznatky, spoločenský dopad, alebo ekonomické výhody.

\end{document}
