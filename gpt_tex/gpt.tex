\documentclass[a4paper,24pt]{article}

\usepackage[utf8]{inputenc}
\usepackage[T1]{fontenc}
\usepackage[slovak]{babel}
\usepackage{geometry}
%\geometry{margin=2.5cm}

% --- Formátovanie ---
\usepackage{setspace}
\usepackage{titlesec}
\setstretch{1.3}

\begin{document}

% --- Mená autorov ---
\begin{center}
    \Large \textbf{Meno Priezvisko, Meno Priezvisko, Meno Priezvisko}
\end{center}

\noindent\rule{\textwidth}{0.4pt} % horizontálna čiara

\vspace{1em}

% --- Názov projektu ---
\begin{center}
    \LARGE \textbf{Rozšírená realita pre interaktívne vzdelávanie}
\end{center}

\vspace{0.5em}

% --- Akronym ---
\begin{center}
    \large \textbf{EduAR}
\end{center}

\vspace{1em}

% --- Úvod, problematika a ciele projektu ---
\section{Úvod, problematika}

Rýchly rozvoj digitálnych technológií a potreba modernizácie vzdelávacieho procesu vytvárajú priestor pre implementáciu inovatívnych nástrojov, ktoré podporujú interaktivitu a angažovanosť študentov. Rozšírená realita (Augmented Reality – AR) umožňuje prepojenie reálneho a virtuálneho sveta, čím poskytuje nové možnosti pre vizualizáciu zložitých javov, experimentov alebo historických udalostí priamo v triede. Napriek rastúcemu potenciálu však chýba ucelený systém využitia AR v školskom prostredí, ktorý by bol pedagogicky efektívny, finančne dostupný a technicky realizovateľný.

Projekt \textbf{EduAR} sa zameriava na výskum, vývoj a overenie metodík využitia rozšírenej reality vo vzdelávaní. Cieľom je prepojiť teoretické poznatky s praktickými aplikáciami v rôznych predmetoch (napr. fyzika, biológia, dejepis) a zvýšiť tak motiváciu, pochopenie a zapamätanie učiva u študentov. Projekt tiež skúma technické a psychologické aspekty adopcie AR nástrojov v školách.

\vspace{2em}
\section{Ciele projektu}

Hlavnými cieľmi projektu sú:
\begin{itemize}
    \item vytvoriť interaktívnu platformu založenú na technológii rozšírenej reality pre podporu výučby,
    \item navrhnúť a overiť didaktické scenáre, ktoré využívajú AR prvky pre rôzne stupne vzdelávania,
    \item preskúmať vplyv využívania AR na motiváciu, porozumenie a zapamätanie učiva,
    \item poskytnúť otvorené metodické materiály pre učiteľov a návrhy integrácie AR do kurikula.
\end{itemize}

Projekt bude realizovaný v úzkej spolupráci s vybranými školami a odborníkmi z oblasti pedagogiky, informatiky a dizajnu používateľského rozhrania.

\vspace{2em}
\section{Výhody}

Očakávaným prínosom projektu je zvýšenie efektivity výučby a zlepšenie študentského zážitku prostredníctvom interaktívneho a zážitkového učenia. AR technológie umožnia vizualizáciu náročných tém, čím sa podporí hlbšie pochopenie učiva. Výsledkom projektu bude vytvorenie nástroja, ktorý môže prispieť k digitalizácii školstva, zvýšeniu kvality výučby a rozvoju digitálnych kompetencií učiteľov aj študentov. Projekt má potenciál priniesť aj spoločenský a ekonomický prínos – zníženie nákladov na fyzické pomôcky, modernizáciu výučby a posilnenie inovačného imidžu slovenského vzdelávacieho systému.

\end{document}
